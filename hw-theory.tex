\documentclass[10pt,a4paper,oneside]{article}
\usepackage[utf8]{inputenc}
\usepackage[english,russian]{babel}
\usepackage{amsmath}
\usepackage{amsthm}
\usepackage{amssymb}
\usepackage{enumerate}
\usepackage{stmaryrd}
\usepackage{cmll}
\usepackage{mathrsfs}
\usepackage[left=2cm,right=2cm,top=2cm,bottom=2cm,bindingoffset=0cm]{geometry}
\usepackage{proof}
\usepackage{tikz}
\usepackage{multicol}
\usepackage{mathabx}

\makeatletter
\newcommand{\dotminus}{\mathbin{\text{\@dotminus}}}

\newcommand{\@dotminus}{%
  \ooalign{\hidewidth\raise1ex\hbox{.}\hidewidth\cr$\m@th-$\cr}%
}
\makeatother

\usetikzlibrary{arrows,backgrounds,patterns,matrix,shapes,fit,calc,shadows,plotmarks}

\newtheorem{definition}{Определение}
\begin{document}

\begin{center}{\Large\textsc{\textbf{Теоретические (``малые'') домашние задания}}}\\
             \it Математическая логика, ИТМО, М3235-М3239, весна 2021 года\end{center}

\section*{Задание №1. Знакомство с исчислением высказываний.}


При решении заданий вам может потребоваться теорема о дедукции (будет доказана на второй лекции): 
$\Gamma, \alpha \vdash \beta$ 
тогда и только тогда, когда $\Gamma \vdash \alpha\rightarrow\beta$. Например, если было показано 
существование вывода $A \vdash A$, то тогда теорема гарантирует и существование вывода $\vdash A \rightarrow A$.

\begin{enumerate}
\item Докажите:
\begin{enumerate}
\item $\vdash (A \rightarrow A \rightarrow B) \rightarrow (A \rightarrow B)$
\item $\vdash \neg (A \with \neg A)$
\item $\vdash A \with B \rightarrow B \with A$
\item $\vdash A \vee B \rightarrow B \vee A$
\item $A \with \neg A \vdash B$
\end{enumerate}

\item Докажите:
\begin{enumerate}
\item $\vdash A \rightarrow \neg \neg A$
\item $\neg A, B \vdash \neg(A\& B)$
\item $\neg A,\neg B \vdash \neg( A\vee B)$
\item $ A,\neg B \vdash \neg( A\rightarrow B)$
\item $\neg A, B \vdash  A\rightarrow B$
\end{enumerate}

\item Докажите:
\begin{enumerate}
\item $\vdash (A \rightarrow B) \rightarrow (B \rightarrow C) \rightarrow (C \rightarrow A)$ 
\item $\vdash (A \rightarrow B) \rightarrow (\neg B \rightarrow \neg A)$ \emph{(правило контрапозиции)}
\item $\vdash A \with B \rightarrow \neg (\neg A \vee \neg B)$
\item $\vdash \neg (\neg A \vee \neg B) \rightarrow (A \with B)$
\item $\vdash (A \rightarrow B) \rightarrow (\neg A \vee B)$
\item $\vdash A \with B \rightarrow A \vee B$
\item $\vdash ((A \rightarrow B) \rightarrow A)\rightarrow A$ \emph{(закон Пирса)}
\end{enumerate}

\item Следует ли какая-нибудь расстановка скобок из другой: $(A \rightarrow B) \rightarrow C$ и 
$A \rightarrow (B \rightarrow C)$? Предложите вывод в исчислении высказываний или докажите, что его не
существует (например, воспользовавшись теоремой о корректности, предложив соответствующую оценку).

\item Предложите схемы аксиом, позволяющие добавить следующие новые связки к исчислению.
\begin{enumerate}
\item связка <<и-не>> (<<штрих шеффера>>, ``|''): $A\ |\ B := \neg (A \with B)$. Новые схемы аксиом должны 
давать возможность исключить конъюнкцию и отрицание из исчисления (то есть, при замене $\neg \alpha$
на $\alpha\ |\ \alpha$ все схемы аксиом для отрицания должны стать теоремами, то же и для конъюнкции).
\item связка <<или-не>> (<<стрелка пирса>>, ``$\downarrow$''): $A \downarrow B := \neg (A \vee B)$.
Новые схемы аксиом должны давать возможность исключить дизъюнкцию и отрицание из исчисления.
\item Нуль-местная связка <<ложь>> (``$\bot$''). Мы ожидаем вот такую замену: $\neg A := A \rightarrow \bot$.
Аналогично, аксиомы для отрицания в новом исчислении должны превратиться в теоремы. 
\end{enumerate}

\item Достаточно ли лжи и <<исключённого или>> ($A \oplus B$ истинно, когда $A \ne B$) для выражения
всех остальных связок? 

\item Даны высказывания $\alpha$ и $\beta$, причём $\vdash \alpha\rightarrow\beta$ и $\not\vdash\beta\rightarrow\alpha$. 
Укажите способ построения высказывания $\gamma$, такого, что
$\vdash\alpha\rightarrow\gamma$ и $\vdash\gamma\rightarrow\beta$, причём $\not\vdash\gamma\rightarrow\alpha$ и
$\not\vdash\beta\rightarrow\gamma$.

\item Покажите, что если $\alpha \vdash \beta$ и $\neg\alpha\vdash\beta$, то $\vdash\beta$.
\end{enumerate}


\end{document}
